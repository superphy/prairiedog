\documentclass{article}

\RequirePackage{hyperref}

\usepackage[parfill]{parskip}
\usepackage[letterpaper, portrait, margin=1in]{geometry}
% Author
\usepackage{authblk}
% Images
\usepackage{graphicx}
\usepackage{lscape}
\usepackage{longtable}
% \usepackage{svg}

%\articlesubtype{This is the article type (optional)}
% \bibliography{paper-webserver.bib}

\usepackage{xcolor}
\newcommand\mwcomment[1]{\textcolor{red}{#1}}

\begin{document}

\title{K-mer based prokaryote pan-genome graph creation for long-term surveillance}

\author[1]{Kevin K Le\thanks{kle009@uottawa.ca}}
\author[1]{Victor PJ Gannon}
\author[2]{Chad R Laing\thanks{chad.laing@canada.ca}}
\affil[1]{National Microbiology Laboratory at Lethbridge, Public Health Agency of Canada, Twp Rd 9-1, Lethbridge, AB, T1J 3Z4, Canada}
\affil[2]{National Centre for Animal Diseases, Lethbridge Laboratory, Canadian Food Inspection Agency, Twp Rd 9-1, Lethbridge, AB, T1J 3Z4, Canada}

\renewcommand\Authands{ and }

\maketitle

\begin{abstract}

Current methods for pan-genome creation, which offer a callable mapping of all genes within a sample population, are based on the identification of known core and accesory genes and produce one-off results which can not be updated as new samples are sequenced.
The lack of an update option is because of compactation steps which reduce the final pangenome size, but prevents individual whole-genome sequence (WGS) samples from being reconstructed from a computed pan-genome.
Here we present a method for pan-genome creation based on k-mers - subsequences of length k - that functions similarly to common pan-genomes, supports updating the pan-genome in linear time, and returns source attribution during variant calling.


\end{abstract}


\section{Introduction}

% text...

% **************************************************************
% Keep this command to avoid text of first page running into the
% first page footnotes
\enlargethispage{-65.1pt}
% **************************************************************

\section{FUNCTIONALITY}


\section{IMPLEMENTATION}

\section{RESULTS}

\section{DISCUSSION}


\section{CONCLUSIONS}


\textit{Conflict of interest}. None declared.

\newpage

\bibliographystyle{unsrt}
\bibliography{paper}

\end{document}
